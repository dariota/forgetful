\epigraph{640K ought to be enough for anybody.}{Not Bill Gates}

Despite the often misattributed epigraph above often being used to mock past beliefs that some amount of memory should be enough for any reasonable purposes, the mentality behind it is still pervasive.

With the broad availability of large amounts of computational power, memory and storage, conservation or efficient use of the same is often overlooked in programming. This is largely perpetuated by the (often valid) view that programmer time is more valuable than the benefits that more efficient but more complex code brings.

However, there remain situations where these benefits are in fact worth the effort required. One of these such cases is in code intended to be deployed in embedded or mobile devices, where resources are limited and preservation of power is essential. \\
A blog post \cite{curlmalloc} by Daniel Stenberg, original author of the \toolname{curl} command line tool and ubiquitous URL data transfer tool, is a retrospective on an attempt to reduce unnecessary heap allocations.

Inspired by that post, the aim of this project is to produce a tool to identify cases where similar changes could be made in order to potentially reduce a program's energy and processing power footprint, and at the same time improve its performance.

\section{Report Structure}

The report is structured as follows:

\begin{itemize}
	\item{Chapter 2 provides background on the project, including further information on the changes to \toolname{cURL} that inspired this project, as well as laying out the objectives of the project}
	\item{Chapter 3 describes the goals of the plugin developed, the platform it built upon, difficulties encountered in development, and the final state of the plugin}
	\item{Chapter 4 covers three case studies, two written intentionally to trigger certain behaviours to maximise the optimisation's effect, and one which simply involves isolating Stenberg's changes and testing their impact}
	\item{Chapter 5 examines the results and outcomes of the case studies and state of the plugin, as well as including a short discussion on potential benefits of future work in this area}
	\item{Chapter 6 describes the state of the art in related areas to the project}
	\item{Chapter 7 describes some areas with potential for future work}
\end{itemize}
